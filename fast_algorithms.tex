\section{Fast Algorithms}

\begin{frame}
    \frametitle{Motivation}

    \begin{enumerate}
        \item solving PDE as BIE leads to small, dense matrices.
        \item Benefits $\rightarrow$ for unbounded problems, integrate over fixed domain, however main tradeoff are dense matrices.
        \item BIE $\rightarrow$ so called fast algorithms, accelerated application and inversion of system matrices that arise upon discretization (Galerkin, Nystrom etc)
        \item 
    \end{enumerate}
\end{frame}

\begin{frame}
    Problem setting for BIE of Helmholtz scattering in exterior.
\end{frame}

\begin{frame}
    \frametitle{History}
    Timeline of fast algorithms, beginning with FMM and latest FDS methods
\end{frame}


\begin{frame}
    \frametitle{FMM - an $\mathcal{O}(N)$ matvec}

    Brief overview of FMM and its underlying operational principle in its analytic form.
\end{frame}

\begin{frame}
    \frametitle{From Analytic to Algebraic Fast algorithms}
    Drawbacks of analytic FMM, and analytic fast algorithms. Give sketch of semi-analytic methods, and what might be accomplished by a fully algebraic method.
\end{frame}


\begin{frame}
    \frametitle{Algebraic Fast Algorithms for Matrix Inversion}
    Fast direct solvers, overview of what they are trying to accomplish, and of course the major pros and cons.
\end{frame}


\begin{frame}
    \frametitle{Summary}
    Summarise the motivation for fast algorithms, and briefly discuss their other applications outside of integral equations.
\end{frame}

